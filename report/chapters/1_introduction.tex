\chapter{Introduction}

Transportation models have grown in complexity by including 
more details on travel behaviour. However, some demographic 
and socioeconomic variables are still considered exogenous 
factors, introducing uncertainty and reducing the model's 
effectiveness in forecasting future scenarios. Since work 
trips are the most frequent trips in urban areas, understanding 
work-related attributes is relevant for travel demand modelling. 
Some attributes such as the place of residence, place of work, 
household income, auto ownership, and mode choice can directly 
or indirectly relate to the labour market.

According to \citet{HarmonMiller2020}, the inclusion of labour 
market interactions within urban simulations has had little 
development despite the critical relationship between work and 
transportation systems in urban areas. In their paper, 
\citet{HarmonMiller2020} proposed an Agent-Based framework for 
simulating the demand and supply interactions of the labour 
market. This framework provided the first approach to a fully 
endogenous labour market simulation within a 
transportation-related model. Nevertheless, as discussed by 
\citet{Harmon2013}, much research still needs to be done to ensure 
the validity of results as the model evolves in time, to 
understand the interactions between agents, and to investigate 
the factors influencing the recruitment process within firms. 
In particular, He provides some evidence that the existent wage 
model could be underestimating salaries in the long run, which 
can be critical given the role of wages in the labour market. 
\\
\\
\\
Therefore, this thesis presents a model that predicts salary 
based on individual attributes of a worker using the principles 
of Bayesian inference. This model improves the prediction 
accuracy by accounting for the hierarchical structure of the 
data, which better simulates the variability in the salaries 
at both an aggregated and disaggregated level.

\section{Outline}

Although all sections in this document are structured 
sequentially, some can be optional according to the reader's 
knowledge of Bayesian inference. After this introduction, 
Chapter 2 presents an overview of economic theory and discusses 
the role of salaries and wages in the labour market interactions. 
Chapter 3 briefly introduces Bayesian inference, the framework 
for estimating the proposed model. Chapter 4 discusses the data 
sources and the hierarchy of labour data. Additionally, it 
presents the main results of the exploratory data analysis that 
guides the model specification. Then, Chapter 5 presents the 
details of the proposed salary model and the validation results 
with new data, followed by Chapter 6, which discusses the 
integration of this proposed model into the existing ILUTE 
framework. Finally, Chapter 7 compiles the principal results, 
and Chapter 8 discusses the future work.