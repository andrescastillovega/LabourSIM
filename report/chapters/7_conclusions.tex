\chapter{Conclusions}

The hierarchical structure in labour markets defines many attributes of both the firms and the individuals. According to the findings in this thesis, this structure is closely related with the observed wage differentials. Therefore, the inclusion of this structure in the modelling process will improve significantly the results and accuracy of the model. However, this structure also needs to consider the relationships between the different levels. The hierarchical or multilevel model is an approach that meets these requirements because it represents the hierarchical structure while also models the complex relationships between the multiple levels by sharing information between them. 

Given the stochastic nature of simulations, the use of a random component seems to be more flexible and realistic than the use of point estimates to represent salaries within the labour market. As salaries are positive a right skewed, the application of the regular linear regression model could be misrepresenting the random component of salary definition. As suggested by the literature, the use of the Gamma distribution in conjunction with a systematic component improves the model performance and allows to capture the particularities of the observed salaries. 

Besides the model structure, the application of the Bayesian inference framework allows to extract more information from the data by producing probability distributions instead of point estimates. This is more evident when modelling out-of-sample salary distributions at both the aggregated and disaggregated level with high accuracy and a good bias-variance balance, which demonstrates its potential to produce robust estimations. However, these advantages come with a cost in terms of computational power. 

\section{Future work}

Based on the results presented in Chapter 5 and Chapter 6, the proposed model needs to be tested within the Harmon's AMB implementation of the labour market in ILUTE. Although Chapter 6 compares the methodologies of the existent and proposed model, applying the last one in a simulation setting will provide evidence whether the use of salary probability distributions outperforms the use of point estimates. 

As the hierarchical model is also applicable to spatial problems, the proposed model could benefit from including job location as a predictor at the industry level. This approach will improve the representation of the decision-making process within the job matching process in ILUTE and provide a more realistic. 

The proposed model is not considering any variation between the model's parameters. However, a possible way to improve the predictive accuracy could be to model the covariance between the predictors by setting a prior over the correlation matrix using the LJK distribution. This distribution provides a way to define a uniform distribution of all possible positive defined correlation matrices. With this approach, some labour market distributions such as the gender and level of education can be better represented for some specific industries and occupations. 

On the other hand, although the SLID dataset is a detailed data source, future efforts should be focused on using more updated datasets to validate the effects of recent changes in the labour force in the estimated models. In this matter, Statistics Canada has an interesting set of statistical programs but most of them require an application process to access this data. Under the light of new data, the proposed model is expected to benefit from the online learning scheme, in which the model can be easily updated under the Bayesian framework. This update just consider that the results of this thesis are the priors to be updated using the new data. 

Another important aspect to consider when evaluating the possible effects of current and future changes in the labour market is the increment of salaries over time. Given the stability observed in most of the salaries during the period of study, the autoregressive nature of the time series could be obviated. However, as the labour markets are continuously changing this behaviour can be different in the future. Therefore, this is something to consider in future updates of this model. 