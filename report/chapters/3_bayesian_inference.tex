% \chapter{An overview of Bayesian inference}

% This section presents an overview of the basic concepts in 
% Bayesian inference that are used for model estimation in the 
% following chapters. It starts with a brief comparison between 
% the Frequentist and Bayesian approaches, then, it presents the 
% theoretical foundations of Bayesian statistics and how these 
% principles are transformed into a practical framework for 
% parameter inference and modelling in the real world.

% This chapter is built on the ideas discussed in 
% \textit{A student's guide to Bayesian statistics} \citep{Lambert2018}, 
% \textit{Statistical rethinking: A Bayesian course with examples in {R}
% and Stan} \citep{McElreath2016}, and 
% \textit{Bayesian Data Analysis} \citep{Gelman2013}. For further 
% details, these sources are a good starting point.

% % ----------------------------------------------------------%
% \section{The Bayesian paradigm: Frequentist vs. Bayesian}

% The goal of statistical inference is to draw conclusions about 
% a population by building a model that better represents the 
% process of interest and estimating the parameters that better 
% describe its behaviour. In statistical inference, there are two 
% \textit{Schools of thought}: the Frequentist (or Classical)  and the 
% Bayesian approach. 

% For frequentists, the data are assumed to be the result of an 
% infinite number of repeated experiments with the same 
% characteristics.  Then, the data are randomly sampled from a 
% fixed and defined population, and any source of variation comes 
% from that sampling process. Under this perspective, model 
% parameters are assumed to be fixed but unknown values related 
% to the population of interest, and the objective of inference 
% is to calculate the best point estimate of the true value of 
% the parameters given a data sample.

% In contrast, Bayesian statistics assume that data are observed 
% and fixed quantities and the source of variation comes from the 
% uncertainty over the parameters. In this view, parameters are 
% probabilistic values and the objective of inference is to 
% estimate the probability distribution of the model’s parameters. 
% Then, we use the data as evidence to update any prior belief 
% about the underlying process.

% The debate about which approach is the best is interesting but 
% long and almost philosophical, therefore, it is out of the 
% scope of this document. However, \citet{Lambert2018} argues 
% that the Frequentist approach might
% make sense if the process of interest can be replicated 
% multiple times as in the case of many natural sciences 
% (i.e. multiple controlled experiments in a laboratory). 
% Conversely, In contexts where the data collection can only be performed 
% once such as in many social sciences, or transportation 
% studies (e.g. democratic elections, population census, 
% travel surveys), a Bayesian approach might be more aligned 
% with the data nature.

% % ----------------------------------------------------------%
% \section{The Bayesian thinking}

% The Bayesian framework shown in eq starts with some prior beliefs about the 
% process we are interested in modelling. Then, 
% we collect evidence (data) to update our beliefs 
% using the model. The result of this update conforms 
% to what is known as posterior belief.

% \begin{equation}\label{eq:bayesian_thinking}
%     \text{prior} + \text{data} \xrightarrow{\ \ \ \ model\ \ \ \ } \text{posterior}
% \end{equation}

% An example of this update procedure in Bayesian inference
%  can be a simple model of the wage gap by gender in a 
%  given labour market. Supported by many researchers and 
%  study results, we can assume there is a gap in salaries 
%  between Males and Females and that this difference is 
%  normally distributed around a positive value. Then, 
%  we collect data from a salary survey and build a linear 
%  model $\hat{y}=\theta\cdot\text{gender}$, assuming the 
%  salaries are linearly correlated with gender. If we have 
%  proper and sufficient data, we obtain the estimated 
%  probability distribution of $\theta$ which corresponds to 
%  the estimated posterior distribution of the salary differences.

% In this example, the location and shape of the posterior
%  distribution of $\theta$ is determined by several 
%  factors: The location and shape of the prior distribution, 
%  the data size and quality, and the model specification.

% Assuming our model is well specified, if we wrongly 
% believe that there is no gender gap in salaries or 
% that the difference is not significant, our data and 
% model will update this belief and produce a posterior 
% closer to the evidence we have. Conversely, If our prior 
% belief and the evidence that we have supports the 
% existence of a gender gap, our process would be well 
% explained by our model and the uncertainty about the 
% process (parameter variability) would be reduced.



% \section{The basics of Bayesian inference}
% \section{Applying the Bayes theorem: Markov Chain Monte Carlo sampling (MCMC)}
% \section{Probabilistic programming: A framework to perform MCMC}